\documentclass[12pt]{article}
\usepackage[utf8]{inputenc}
\usepackage[colorlinks]{hyperref}
\usepackage{cite}

\title{Literature Review: CSE 6730 Project 1}
\author{Chris Dunlap, Allen Koh, Matt May}
\date{February 2016}

\begin{document}

\begin{titlepage}
\maketitle
\thispagestyle{empty}
\end{titlepage}

A variety of techniques have been used to model pedestrian movement. Cellular
automata, lattice gas, social force, fluid-dynamic, agent-based, game-theoretic,
and animal experimention-based approaches have been used
\cite{zheng2009modeling}. As noted by Zheng, Zhong, and Liu
\cite{zheng2009modeling}, models often encounter or attempt to model several
common phenomenoma: clogging, side-stepping, lane formation, and herding
behavior, among them.

These phenomena manifest themselves in different ways and at different magnitudes
depending on the system being modeled. In one of the early explorations of 2D
cellular automata to simulating vehicular traffic flow, Biham and Middleton
\cite{biham1992self} note the presence of a sharp \textit{jamming transition}
in which all cars in the simulation transition from moving at maximal speed to
being stuck. Similar effects were noted in simulations of bi-directional
pedestrian movement, with Weifeng, Lizhong, and Weicheng
\cite{weifeng2003simulation} finding that as total pedestrian density increases,
a critical value is reached at which the system transitions into a jammed state
where only a few pedestrians are able to proceed.

In a slightly different take on the problem, Okazaki and Matsushita
\cite{okazaki1993study} modeled pedestrian evacuation using Coulomb's Law, with
actors magnetically moving toward their goals and away from obstacles that would
lead to collisions. In another application of equations of natural phenomena,
Helbing \cite{helbing1998fluid} derived fluid dynamic equations for explaining
the movement of pedestrian crowds, observing phenomena such as the development
of lanes, jamming, and crossing. Helbing \cite{helbing2000simulating} further
notes an explicit ``faster-is-slower" effect, in which attempting to move faster
results in a smaller average speed of leaving at high pedestrian density in
evacuation scenarios.

In a more recent study, Helbing et al. \cite{helbing2005self} uses a
``social force" model, which states that pedestrians operate in some sense
automatically when reacting to obstacles and other pedestrians, applying
strategies that have been learned to be most effective over time. Several
suggestions are made to alleviate the three most common problems in pedestrian
crowds: counterflows, bottlenecks, and intersecting flows. The presence of
strategically placed obstacles are found to actually reduce these negative
phenomena, leading to improved flow.

Building on previous cellular automata-based models, Burstedde et al.
\cite{burstedde2001simulation} introduce a \textit{floor field}, a secondary
grid of cells which underlies the main grid and acts as a substitute for
pedestrian intelligence. These fields can be either static or dynamic, and are
capable of promoting the avoidance of jams, as well as simulating attractive
effects, in which pedestrians are more likely to follow in the paths of
pedestrians ahead of them.

Taking a more algorithmic approach, Fang et al. \cite{fang2011hierarchical} have
designed a modified ant colony optimization (ACO) algorithm for optimizing
pedestrian evacuation, seeking to minimize evacuation time, evacuation distance,
and congestion. Kemloh Wagoum, Seyfried, and Holl \cite{kemloh2012modeling}
utilize an observation principle approach in modeling pedestrian evacuation,
in which pedestrians first observe their environment and then make a final
decision on strategy based on obtained data.







\bibliography{template}{}
\bibliographystyle{plain}
\end{document}